% Options for packages loaded elsewhere
\PassOptionsToPackage{unicode}{hyperref}
\PassOptionsToPackage{hyphens}{url}
%
\documentclass[
]{article}
\usepackage{lmodern}
\usepackage{amssymb,amsmath}
\usepackage{ifxetex,ifluatex}
\ifnum 0\ifxetex 1\fi\ifluatex 1\fi=0 % if pdftex
  \usepackage[T1]{fontenc}
  \usepackage[utf8]{inputenc}
  \usepackage{textcomp} % provide euro and other symbols
\else % if luatex or xetex
  \usepackage{unicode-math}
  \defaultfontfeatures{Scale=MatchLowercase}
  \defaultfontfeatures[\rmfamily]{Ligatures=TeX,Scale=1}
\fi
% Use upquote if available, for straight quotes in verbatim environments
\IfFileExists{upquote.sty}{\usepackage{upquote}}{}
\IfFileExists{microtype.sty}{% use microtype if available
  \usepackage[]{microtype}
  \UseMicrotypeSet[protrusion]{basicmath} % disable protrusion for tt fonts
}{}
\makeatletter
\@ifundefined{KOMAClassName}{% if non-KOMA class
  \IfFileExists{parskip.sty}{%
    \usepackage{parskip}
  }{% else
    \setlength{\parindent}{0pt}
    \setlength{\parskip}{6pt plus 2pt minus 1pt}}
}{% if KOMA class
  \KOMAoptions{parskip=half}}
\makeatother
\usepackage{xcolor}
\IfFileExists{xurl.sty}{\usepackage{xurl}}{} % add URL line breaks if available
\IfFileExists{bookmark.sty}{\usepackage{bookmark}}{\usepackage{hyperref}}
\hypersetup{
  pdftitle={Gene expression of L. plantarum WCFS1},
  pdfauthor={Ivar van den Akker, Herke Schuffel, Yuri Wit en Peter Cserei},
  hidelinks,
  pdfcreator={LaTeX via pandoc}}
\urlstyle{same} % disable monospaced font for URLs
\usepackage[margin=1in]{geometry}
\usepackage{color}
\usepackage{fancyvrb}
\newcommand{\VerbBar}{|}
\newcommand{\VERB}{\Verb[commandchars=\\\{\}]}
\DefineVerbatimEnvironment{Highlighting}{Verbatim}{commandchars=\\\{\}}
% Add ',fontsize=\small' for more characters per line
\usepackage{framed}
\definecolor{shadecolor}{RGB}{248,248,248}
\newenvironment{Shaded}{\begin{snugshade}}{\end{snugshade}}
\newcommand{\AlertTok}[1]{\textcolor[rgb]{0.94,0.16,0.16}{#1}}
\newcommand{\AnnotationTok}[1]{\textcolor[rgb]{0.56,0.35,0.01}{\textbf{\textit{#1}}}}
\newcommand{\AttributeTok}[1]{\textcolor[rgb]{0.77,0.63,0.00}{#1}}
\newcommand{\BaseNTok}[1]{\textcolor[rgb]{0.00,0.00,0.81}{#1}}
\newcommand{\BuiltInTok}[1]{#1}
\newcommand{\CharTok}[1]{\textcolor[rgb]{0.31,0.60,0.02}{#1}}
\newcommand{\CommentTok}[1]{\textcolor[rgb]{0.56,0.35,0.01}{\textit{#1}}}
\newcommand{\CommentVarTok}[1]{\textcolor[rgb]{0.56,0.35,0.01}{\textbf{\textit{#1}}}}
\newcommand{\ConstantTok}[1]{\textcolor[rgb]{0.00,0.00,0.00}{#1}}
\newcommand{\ControlFlowTok}[1]{\textcolor[rgb]{0.13,0.29,0.53}{\textbf{#1}}}
\newcommand{\DataTypeTok}[1]{\textcolor[rgb]{0.13,0.29,0.53}{#1}}
\newcommand{\DecValTok}[1]{\textcolor[rgb]{0.00,0.00,0.81}{#1}}
\newcommand{\DocumentationTok}[1]{\textcolor[rgb]{0.56,0.35,0.01}{\textbf{\textit{#1}}}}
\newcommand{\ErrorTok}[1]{\textcolor[rgb]{0.64,0.00,0.00}{\textbf{#1}}}
\newcommand{\ExtensionTok}[1]{#1}
\newcommand{\FloatTok}[1]{\textcolor[rgb]{0.00,0.00,0.81}{#1}}
\newcommand{\FunctionTok}[1]{\textcolor[rgb]{0.00,0.00,0.00}{#1}}
\newcommand{\ImportTok}[1]{#1}
\newcommand{\InformationTok}[1]{\textcolor[rgb]{0.56,0.35,0.01}{\textbf{\textit{#1}}}}
\newcommand{\KeywordTok}[1]{\textcolor[rgb]{0.13,0.29,0.53}{\textbf{#1}}}
\newcommand{\NormalTok}[1]{#1}
\newcommand{\OperatorTok}[1]{\textcolor[rgb]{0.81,0.36,0.00}{\textbf{#1}}}
\newcommand{\OtherTok}[1]{\textcolor[rgb]{0.56,0.35,0.01}{#1}}
\newcommand{\PreprocessorTok}[1]{\textcolor[rgb]{0.56,0.35,0.01}{\textit{#1}}}
\newcommand{\RegionMarkerTok}[1]{#1}
\newcommand{\SpecialCharTok}[1]{\textcolor[rgb]{0.00,0.00,0.00}{#1}}
\newcommand{\SpecialStringTok}[1]{\textcolor[rgb]{0.31,0.60,0.02}{#1}}
\newcommand{\StringTok}[1]{\textcolor[rgb]{0.31,0.60,0.02}{#1}}
\newcommand{\VariableTok}[1]{\textcolor[rgb]{0.00,0.00,0.00}{#1}}
\newcommand{\VerbatimStringTok}[1]{\textcolor[rgb]{0.31,0.60,0.02}{#1}}
\newcommand{\WarningTok}[1]{\textcolor[rgb]{0.56,0.35,0.01}{\textbf{\textit{#1}}}}
\usepackage{graphicx,grffile}
\makeatletter
\def\maxwidth{\ifdim\Gin@nat@width>\linewidth\linewidth\else\Gin@nat@width\fi}
\def\maxheight{\ifdim\Gin@nat@height>\textheight\textheight\else\Gin@nat@height\fi}
\makeatother
% Scale images if necessary, so that they will not overflow the page
% margins by default, and it is still possible to overwrite the defaults
% using explicit options in \includegraphics[width, height, ...]{}
\setkeys{Gin}{width=\maxwidth,height=\maxheight,keepaspectratio}
% Set default figure placement to htbp
\makeatletter
\def\fps@figure{htbp}
\makeatother
\setlength{\emergencystretch}{3em} % prevent overfull lines
\providecommand{\tightlist}{%
  \setlength{\itemsep}{0pt}\setlength{\parskip}{0pt}}
\setcounter{secnumdepth}{-\maxdimen} % remove section numbering

\title{Gene expression of L. plantarum WCFS1}
\author{Ivar van den Akker, Herke Schuffel, Yuri Wit en Peter Cserei}
\date{11/24/2020}

\begin{document}
\maketitle

{
\setcounter{tocdepth}{4}
\tableofcontents
}
\hypertarget{samenvatting}{%
\subsection{Samenvatting}\label{samenvatting}}

\hypertarget{inleiding}{%
\subsection{Inleiding}\label{inleiding}}

Doel: When L. plantarum is grown on a ribose-rich medium genes are
upregulated that are required for metabolizing ribose.

\hypertarget{materiaal-en-methode}{%
\subsection{Materiaal en Methode}\label{materiaal-en-methode}}

Dit R-script analyseert de expressie van genen van L. plantarum
varianten: WCFS1 en NC8. Deze twee bacterien zijn gegroeid op
verschillende bodems namelijk glucose (glc) en ribose (rib).

\hypertarget{library-importeren}{%
\paragraph{Library importeren}\label{library-importeren}}

Importeren van library's die nodig zijn voor het functioneren van
onderstaande script.

\begin{Shaded}
\begin{Highlighting}[]
\KeywordTok{library}\NormalTok{(limma)}
\KeywordTok{library}\NormalTok{(edgeR)}
\KeywordTok{library}\NormalTok{(dplyr)}
\end{Highlighting}
\end{Shaded}

\begin{verbatim}
## 
## Attaching package: 'dplyr'
\end{verbatim}

\begin{verbatim}
## The following objects are masked from 'package:stats':
## 
##     filter, lag
\end{verbatim}

\begin{verbatim}
## The following objects are masked from 'package:base':
## 
##     intersect, setdiff, setequal, union
\end{verbatim}

\begin{Shaded}
\begin{Highlighting}[]
\KeywordTok{library}\NormalTok{(xlsx)}
\KeywordTok{library}\NormalTok{(factoextra)}
\end{Highlighting}
\end{Shaded}

\begin{verbatim}
## Loading required package: ggplot2
\end{verbatim}

\begin{verbatim}
## Welcome! Want to learn more? See two factoextra-related books at https://goo.gl/ve3WBa
\end{verbatim}

\begin{Shaded}
\begin{Highlighting}[]
\KeywordTok{library}\NormalTok{(cluster)}
\end{Highlighting}
\end{Shaded}

\hypertarget{bestand-inladen}{%
\paragraph{Bestand inladen}\label{bestand-inladen}}

Het eerste variabel slaat de locatie van het bestand op. Vanzelfsprekend
wordt ook de naam van het nodige bestand in een variabel opgeslagen.

De read.delim functie wordt over het algemeen gebruikt om tekstbestanden
in te lezen waarbij data georganiseerd is in een data matrix layout.
Binnen deze functie zijn de bovenste twee variabelen aan elkaar te
plakken met paste0.

Daarbij wordt het kolom dat zorgt voor het oplopende nummering van rijen
omgezet naar de ID kolom. Dit is gedaan om te voorkomen dat er
verwarring ontstaat. De nummering van R zou namelijk niet overeen komen
met de ID's van de ingeladen data. Dit is van belang voor eventuele
latere annotatie.

\begin{Shaded}
\begin{Highlighting}[]
\NormalTok{fDir <-}\StringTok{ "./"}
\NormalTok{fName <-}\StringTok{ "RNA-Seq-counts.txt"}
\NormalTok{cnts <-}\StringTok{ }\KeywordTok{read.delim}\NormalTok{(}\KeywordTok{paste0}\NormalTok{(fDir,fName), }\DataTypeTok{comment.char=}\StringTok{"#"}\NormalTok{)}
\KeywordTok{row.names}\NormalTok{(cnts) <-}\StringTok{ }\NormalTok{cnts[,}\StringTok{"ID"}\NormalTok{]}

\NormalTok{f2Name <-}\StringTok{ "WCFS1_anno.txt"}
\NormalTok{annotation <-}\StringTok{ }\KeywordTok{read.delim}\NormalTok{(}\KeywordTok{paste0}\NormalTok{(fDir,f2Name), }\DataTypeTok{comment.char =} \StringTok{"#"}\NormalTok{)}
\KeywordTok{row.names}\NormalTok{(annotation) <-}\StringTok{ }\NormalTok{annotation[,}\KeywordTok{row_number}\NormalTok{(}\DecValTok{0}\NormalTok{)]}
\end{Highlighting}
\end{Shaded}

\hypertarget{dge-list-maken}{%
\paragraph{DGE list maken}\label{dge-list-maken}}

EdgeR slaat data op in een list-based data object genaamd een DGEList.
Dit type object is gemakkelijk te gebruiken binnen R aangezien een
DGEList gemanipuleerd kan worden zoals ieder ander lijst in R. Daarbij
maakt een DGElist het mogelijk om berekeningen en statistiek toe te
passen op de data. In het script worden de counts in de kolommen 2 t/m 9
gegroepeerd met de acht aangemaakte labels.

\begin{Shaded}
\begin{Highlighting}[]
\NormalTok{label <-}\StringTok{ }\KeywordTok{c}\NormalTok{(}\StringTok{"WCFS1.glc"}\NormalTok{,  }\StringTok{"WCFS1.glc"}\NormalTok{,   }\StringTok{"WCFS1.rib"}\NormalTok{,    }\StringTok{"WCFS1.rib"}\NormalTok{,    }\StringTok{"NC8.glc"}\NormalTok{,  }\StringTok{"NC8.glc"}\NormalTok{,  }\StringTok{"NC8.rib"}\NormalTok{,  }\StringTok{"NC8.rib"}\NormalTok{)}
\NormalTok{group <-}\StringTok{ }\KeywordTok{factor}\NormalTok{(label)}
\NormalTok{y <-}\StringTok{ }\KeywordTok{DGEList}\NormalTok{(}\DataTypeTok{counts=}\NormalTok{cnts[,}\DecValTok{2}\OperatorTok{:}\DecValTok{9}\NormalTok{],}\DataTypeTok{group=}\NormalTok{group)}
\end{Highlighting}
\end{Shaded}

\hypertarget{data-normaliseren}{%
\paragraph{Data normaliseren}\label{data-normaliseren}}

Normalisatie wordt toegepast om biologische variatie uit de data te
halen, zodat vervolgens de verschillende datasets met elkaar vergeleken
kunnen worden. In R wordt dit gedaan door norm factors te berekenen. De
TMM (trimmed mean of M-values) methode verwijdert eerst de laagste en
hoogste waarden en bepaalt daarna het gemiddelde van de rest van de
genen.

Hieronder zijn de norm factors weergegeven, zonder calcNormFactors toe
te passen zou deze voor alle rijen 1 zijn. De data is genormaliseerd
nadat het met de norm factors vermenigvuldigd is. Dit proces gaat
automatisch in R na het veranderen van de norm factors.

\begin{Shaded}
\begin{Highlighting}[]
\NormalTok{y <-}\StringTok{ }\KeywordTok{calcNormFactors}\NormalTok{(y, }\DataTypeTok{method=}\StringTok{"TMM"}\NormalTok{ )}
\KeywordTok{print}\NormalTok{(y}\OperatorTok{$}\NormalTok{samples)}
\end{Highlighting}
\end{Shaded}

\begin{verbatim}
##                 group lib.size norm.factors
## WCFS1.glc.1 WCFS1.glc 10153710    0.9850714
## WCFS1.glc.2 WCFS1.glc 10615392    0.9830048
## WCFS1.rib.1 WCFS1.rib  8959060    0.9986438
## WCFS1.rib.2 WCFS1.rib  9340139    1.0375329
## NC8.glc.1     NC8.glc  9821662    0.9390775
## NC8.glc.2     NC8.glc 10261922    1.0155475
## NC8.rib.1     NC8.rib  9951954    0.9899344
## NC8.rib.2     NC8.rib  9957519    1.0557378
\end{verbatim}

\hypertarget{filteren-op-low-counts}{%
\paragraph{Filteren op low counts}\label{filteren-op-low-counts}}

Deze functie zorgt ervoor dat genen geselecteert worden die niet
significant genoeg zijn voor de dataset. Aangezien de dataset waarmee
gewerkt wordt al een keer is gefilterd, resulteert de functie ``filter
by expression'' maar in een hele kleine hoeveelheid genen, namelijk 20
stuks.

\begin{Shaded}
\begin{Highlighting}[]
\NormalTok{keep <-}\StringTok{ }\KeywordTok{filterByExpr}\NormalTok{(y)}
\NormalTok{y <-}\StringTok{ }\NormalTok{y[keep,]}
\end{Highlighting}
\end{Shaded}

\hypertarget{design-matrix-maken}{%
\paragraph{Design matrix maken}\label{design-matrix-maken}}

Een design matrix geeft aan hoe de samples gegroepeerd zijn in de
experimenten. In dit geval is te zien dat glucose 1 en 2 gekoppeld zijn
aan glucose van het desbetreffende bacterie. Hetzelfde geldt voor
ribose.

\begin{Shaded}
\begin{Highlighting}[]
\NormalTok{design <-}\StringTok{ }\KeywordTok{model.matrix}\NormalTok{(}\OperatorTok{~}\DecValTok{0}\OperatorTok{+}\NormalTok{group, }\DataTypeTok{data=}\NormalTok{y}\OperatorTok{$}\NormalTok{samples)}
\KeywordTok{colnames}\NormalTok{(design) <-}\StringTok{ }\KeywordTok{levels}\NormalTok{(y}\OperatorTok{$}\NormalTok{samples}\OperatorTok{$}\NormalTok{group)}
\KeywordTok{print}\NormalTok{(design)}
\end{Highlighting}
\end{Shaded}

\begin{verbatim}
##             NC8.glc NC8.rib WCFS1.glc WCFS1.rib
## WCFS1.glc.1       0       0         1         0
## WCFS1.glc.2       0       0         1         0
## WCFS1.rib.1       0       0         0         1
## WCFS1.rib.2       0       0         0         1
## NC8.glc.1         1       0         0         0
## NC8.glc.2         1       0         0         0
## NC8.rib.1         0       1         0         0
## NC8.rib.2         0       1         0         0
## attr(,"assign")
## [1] 1 1 1 1
## attr(,"contrasts")
## attr(,"contrasts")$group
## [1] "contr.treatment"
\end{verbatim}

\hypertarget{despersie-berekenen}{%
\paragraph{Despersie berekenen}\label{despersie-berekenen}}

Er zijn twee verschillende manieren getest om de dispersie te berekenen.

\begin{Shaded}
\begin{Highlighting}[]
\NormalTok{y <-}\StringTok{ }\KeywordTok{estimateDisp}\NormalTok{(y, design)}

\NormalTok{f <-}\StringTok{ }\KeywordTok{estimateGLMCommonDisp}\NormalTok{(y,design)}
\NormalTok{f <-}\StringTok{ }\KeywordTok{estimateGLMTrendedDisp}\NormalTok{(y,design, }\DataTypeTok{method=}\StringTok{"power"}\NormalTok{)}
\NormalTok{f <-}\StringTok{ }\KeywordTok{estimateGLMTagwiseDisp}\NormalTok{(y,design)}
\end{Highlighting}
\end{Shaded}

\hypertarget{fit-data}{%
\paragraph{Fit data}\label{fit-data}}

De glmFit functie maakt een DGEGLM object aan waarin allerlei data wordt
verwerkt en berekent daarbij onder andere de p-values en fold-changes.

\begin{Shaded}
\begin{Highlighting}[]
\NormalTok{fit <-}\StringTok{ }\KeywordTok{glmFit}\NormalTok{(y,design)}
\end{Highlighting}
\end{Shaded}

\hypertarget{determine-fold-changes}{%
\paragraph{Determine fold changes}\label{determine-fold-changes}}

\begin{Shaded}
\begin{Highlighting}[]
\NormalTok{WCFSglcrib <-}\StringTok{ }\KeywordTok{makeContrasts}\NormalTok{(}\DataTypeTok{exp.r=}\NormalTok{WCFS1.glc}\OperatorTok{-}\NormalTok{WCFS1.rib, }\DataTypeTok{levels=}\NormalTok{design)}
\NormalTok{WCFSglcribfit <-}\StringTok{ }\KeywordTok{glmLRT}\NormalTok{(fit, }\DataTypeTok{contrast=}\NormalTok{WCFSglcrib)}
\NormalTok{NC8glcrib <-}\StringTok{ }\KeywordTok{makeContrasts}\NormalTok{(}\DataTypeTok{exp.r=}\NormalTok{NC8.glc}\OperatorTok{-}\NormalTok{NC8.rib, }\DataTypeTok{levels=}\NormalTok{design)}
\NormalTok{NC8glcribfit <-}\StringTok{ }\KeywordTok{glmLRT}\NormalTok{(fit, }\DataTypeTok{contrast=}\NormalTok{NC8glcrib)}
\end{Highlighting}
\end{Shaded}

\hypertarget{print-top-tags}{%
\paragraph{Print top tags}\label{print-top-tags}}

WCFS glucose vs WCFS ribose

\begin{Shaded}
\begin{Highlighting}[]
\NormalTok{WCFSglcribres<-}\KeywordTok{topTags}\NormalTok{(WCFSglcribfit, }\DataTypeTok{n=}\KeywordTok{nrow}\NormalTok{(WCFSglcribfit))}
\end{Highlighting}
\end{Shaded}

NC8 glucose vs NC8 ribose

\begin{Shaded}
\begin{Highlighting}[]
\NormalTok{NC8glcribres<-}\KeywordTok{topTags}\NormalTok{(NC8glcribfit, }\DataTypeTok{n=}\KeywordTok{nrow}\NormalTok{(NC8glcribfit))}
\end{Highlighting}
\end{Shaded}

\hypertarget{filteren}{%
\paragraph{Filteren}\label{filteren}}

Om genen te selecteren die voor ons onderzoek van toepassing zijn hebben
we cut of values ingesteld voor de p-value en de fold changes. Voor de
p-value is er gekozen om deze onder de 0.05 te houden en de fold changes
moeten hoger zijn dan 1 of lager zijn dan -1. WCFS glucose vs WCFS
ribose

\begin{Shaded}
\begin{Highlighting}[]
\NormalTok{WCFSfinal <-}\StringTok{ }\KeywordTok{filter}\NormalTok{(WCFSglcribres}\OperatorTok{$}\NormalTok{table,  WCFSglcribres}\OperatorTok{$}\NormalTok{table}\OperatorTok{$}\NormalTok{FDR}\OperatorTok{<}\FloatTok{0.005} \OperatorTok{&}\NormalTok{(WCFSglcribres}\OperatorTok{$}\NormalTok{table}\OperatorTok{$}\NormalTok{logFC}\OperatorTok{>}\DecValTok{1}\OperatorTok{|}\NormalTok{WCFSglcribres}\OperatorTok{$}\NormalTok{table}\OperatorTok{$}\NormalTok{logFC}\OperatorTok{<}\NormalTok{(}\OperatorTok{-}\DecValTok{1}\NormalTok{)))}
\end{Highlighting}
\end{Shaded}

NC8 glucose vs NC8 ribose

\begin{Shaded}
\begin{Highlighting}[]
\NormalTok{NC8final <-}\StringTok{ }\KeywordTok{filter}\NormalTok{(NC8glcribres}\OperatorTok{$}\NormalTok{table,  NC8glcribres}\OperatorTok{$}\NormalTok{table}\OperatorTok{$}\NormalTok{FDR}\OperatorTok{<}\FloatTok{0.005} \OperatorTok{&}\NormalTok{(NC8glcribres}\OperatorTok{$}\NormalTok{table}\OperatorTok{$}\NormalTok{logFC}\OperatorTok{>}\DecValTok{1}\OperatorTok{|}\NormalTok{NC8glcribres}\OperatorTok{$}\NormalTok{table}\OperatorTok{$}\NormalTok{logFC}\OperatorTok{<}\NormalTok{(}\OperatorTok{-}\DecValTok{1}\NormalTok{)))}
\end{Highlighting}
\end{Shaded}

Annotatie toevoegen aan de genen die zijn gefiltered en van hoog naar
laag sorteren

\begin{Shaded}
\begin{Highlighting}[]
\NormalTok{WCFS1_sig <-}\StringTok{ }\KeywordTok{cbind}\NormalTok{(WCFSfinal, annotation[}\KeywordTok{rownames}\NormalTok{(WCFSfinal),])}
\NormalTok{NC8_sig <-}\StringTok{ }\KeywordTok{cbind}\NormalTok{(NC8final, annotation[}\KeywordTok{rownames}\NormalTok{(NC8final),])}
\NormalTok{WCFS1_sorted <-}\StringTok{ }\NormalTok{WCFS1_sig[}\KeywordTok{order}\NormalTok{(WCFS1_sig[, }\StringTok{"logFC"}\NormalTok{], }\DataTypeTok{decreasing =} \OtherTok{TRUE}\NormalTok{, }\DataTypeTok{na.last =} \OtherTok{FALSE}\NormalTok{), ,drop=}\OtherTok{FALSE}\NormalTok{]}
\NormalTok{NC8_sorted <-}\StringTok{ }\NormalTok{NC8_sig[}\KeywordTok{order}\NormalTok{(NC8_sig[, }\StringTok{"logFC"}\NormalTok{], }\DataTypeTok{decreasing =} \OtherTok{TRUE}\NormalTok{, }\DataTypeTok{na.last =} \OtherTok{FALSE}\NormalTok{), ,drop=}\OtherTok{FALSE}\NormalTok{]}
\end{Highlighting}
\end{Shaded}

Excel file maken met de data en annotaties

\begin{Shaded}
\begin{Highlighting}[]
\CommentTok{#write.xlsx(WCFS1_sorted, file = "Merged_Genes.xlsx", sheetName = "WCFS1_data", }
\CommentTok{#  col.names = TRUE, row.names = TRUE, append = FALSE)}
\CommentTok{#write.xlsx(NC8_sorted, file = "Merged_Genes.xlsx", sheetName = "NC8_data", }
\CommentTok{#  col.names = TRUE, row.names = TRUE, append = TRUE)}
\end{Highlighting}
\end{Shaded}

\hypertarget{kegg-mapper}{%
\paragraph{KEGG Mapper}\label{kegg-mapper}}

\begin{Shaded}
\begin{Highlighting}[]
\NormalTok{WCFS1_final_names <-}\StringTok{ }\KeywordTok{c}\NormalTok{(}\KeywordTok{rownames}\NormalTok{(}\KeywordTok{head}\NormalTok{(WCFS1_sorted, }\DecValTok{5}\NormalTok{)), }\KeywordTok{rownames}\NormalTok{(}\KeywordTok{tail}\NormalTok{(WCFS1_sorted, }\DecValTok{5}\NormalTok{)))}
\NormalTok{WCFS1<-}\KeywordTok{data.frame}\NormalTok{(WCFS1_final_names)}
\NormalTok{WCFS1}\OperatorTok{$}\NormalTok{logFC<-}\KeywordTok{c}\NormalTok{((}\KeywordTok{head}\NormalTok{(WCFS1_sorted, }\DecValTok{5}\NormalTok{)}\OperatorTok{$}\NormalTok{logFC), (}\KeywordTok{tail}\NormalTok{(WCFS1_sorted, }\DecValTok{5}\NormalTok{)}\OperatorTok{$}\NormalTok{logFC))}
\NormalTok{NC8_final_names <-}\StringTok{ }\KeywordTok{c}\NormalTok{(}\KeywordTok{rownames}\NormalTok{(}\KeywordTok{head}\NormalTok{(NC8_sorted, }\DecValTok{5}\NormalTok{)), }\KeywordTok{rownames}\NormalTok{(}\KeywordTok{tail}\NormalTok{(NC8_sorted, }\DecValTok{5}\NormalTok{)))}
\NormalTok{NC8<-}\KeywordTok{data.frame}\NormalTok{(NC8_final_names)}
\NormalTok{NC8}\OperatorTok{$}\NormalTok{logFC<-}\KeywordTok{c}\NormalTok{((}\KeywordTok{head}\NormalTok{(NC8_sorted, }\DecValTok{5}\NormalTok{)}\OperatorTok{$}\NormalTok{logFC), (}\KeywordTok{tail}\NormalTok{(NC8_sorted, }\DecValTok{5}\NormalTok{)}\OperatorTok{$}\NormalTok{logFC))}
\end{Highlighting}
\end{Shaded}

\hypertarget{heat-map}{%
\paragraph{Heat Map}\label{heat-map}}

\begin{Shaded}
\begin{Highlighting}[]
\CommentTok{# Sorry ik was verkeerde aan het doen mischien heb je er wel wat aan}
\NormalTok{HMNC8 =}\StringTok{ }\KeywordTok{data.matrix}\NormalTok{(NC8final }\OperatorTok\StringTok{ }\KeywordTok{select}\NormalTok{(}\OperatorTok{-}\NormalTok{(LR}\OperatorTok{:}\NormalTok{FDR)))}
\KeywordTok{heatmap}\NormalTok{(HMNC8)}
\end{Highlighting}
\end{Shaded}

\includegraphics{Script_files/figure-latex/unnamed-chunk-16-1.pdf}

\hypertarget{resultaten}{%
\subsection{Resultaten}\label{resultaten}}

\hypertarget{cluster-genes-using-hierarchal-clustering}{%
\paragraph{Cluster genes using hierarchal
clustering}\label{cluster-genes-using-hierarchal-clustering}}

\begin{Shaded}
\begin{Highlighting}[]
\NormalTok{x <-}\StringTok{ }\KeywordTok{t}\NormalTok{(y}\OperatorTok{$}\NormalTok{counts)}
\NormalTok{x <-}\StringTok{ }\KeywordTok{dist}\NormalTok{(x, }\DataTypeTok{method =} \StringTok{"euclidean"}\NormalTok{)}
\NormalTok{x <-}\StringTok{ }\KeywordTok{hclust}\NormalTok{(x, }\DataTypeTok{method =} \StringTok{"average"}\NormalTok{)}
\KeywordTok{plot}\NormalTok{(x)}
\end{Highlighting}
\end{Shaded}

\includegraphics{Script_files/figure-latex/unnamed-chunk-17-1.pdf}

\hypertarget{pca-plot}{%
\paragraph{PCA plot}\label{pca-plot}}

De PCA plot (principal component analysis) comprimeert de datasets door
de data zo veel mogelijk om te zetten naar de 1ste dimensies. Het
verminderen van dimensies van de variatie in de samples maakt het
mogelijk om de ze gemakkelijk te visualiseren en met elkaar te
vergelijken. De X en Y as geven de fold-change in de log2 schaal weer.
De X-as verklaart de hoogste variatie, in dit geval laat het zien dat
het verschil in fold-change tussen WCFS1.glucose en WCFS1.ribose zeer
groot is. Daarbij valt ook op dat de groepen die in de design matrix
gemaakt zijn weinig variatie met elkaar hebben.

\begin{Shaded}
\begin{Highlighting}[]
\KeywordTok{plotMDS}\NormalTok{(y)}
\end{Highlighting}
\end{Shaded}

\includegraphics{Script_files/figure-latex/unnamed-chunk-18-1.pdf}

\hypertarget{dispersie-plot}{%
\paragraph{Dispersie plot}\label{dispersie-plot}}

De dispersie plot laat de spreiding van de data zien. Dit zegt iets over
hoe goed de verschillende samples met elkaar te vergelijken zijn. Op de
X-as staat de CPM (counts per million) op de log schaal en Y-as de
dispersie. De rode lijn is de common, het ideale scenario. De blauwe
lijn laat de afwijking zien van variatie van de verschillende samples.
Hier is te zien dat de trendlijn redelijk overeen komt met de common
lijn.

\begin{Shaded}
\begin{Highlighting}[]
\KeywordTok{plotBCV}\NormalTok{(y)}
\end{Highlighting}
\end{Shaded}

\includegraphics{Script_files/figure-latex/unnamed-chunk-19-1.pdf}

\begin{Shaded}
\begin{Highlighting}[]
\KeywordTok{plotBCV}\NormalTok{(f)}
\end{Highlighting}
\end{Shaded}

\includegraphics{Script_files/figure-latex/unnamed-chunk-19-2.pdf}

\hypertarget{k-means}{%
\paragraph{K-means}\label{k-means}}

\begin{Shaded}
\begin{Highlighting}[]
\NormalTok{km <-}\StringTok{ }\KeywordTok{kmeans}\NormalTok{((NC8final }\OperatorTok\StringTok{ }\KeywordTok{select}\NormalTok{(}\OperatorTok{-}\NormalTok{(logCPM}\OperatorTok{:}\NormalTok{FDR))), }\DataTypeTok{centers =} \DecValTok{5}\NormalTok{)}
\KeywordTok{fviz_cluster}\NormalTok{(km, }\DataTypeTok{data =}\NormalTok{ (NC8final }\OperatorTok\StringTok{ }\KeywordTok{select}\NormalTok{(}\OperatorTok{-}\NormalTok{(LR}\OperatorTok{:}\NormalTok{FDR))))}
\end{Highlighting}
\end{Shaded}

\includegraphics{Script_files/figure-latex/unnamed-chunk-20-1.pdf} \#\#
Conclusie

\end{document}
